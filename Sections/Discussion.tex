\section{Discussion}
"Nothing is greater than its weakest link". This is how to physical model can be described. Even with several tests optimizing steps, the input from the IR-sensor was noisy. This makes it really hard to regulate the ball without any small movements. To get at quick responding regulator for this application the D-term is mandatory, and also with a high gain. However, high gain on the D-term will make the system more responsible to noise. Using digital filters, such as median filter or Kalman filter, the noise was reduced, but filtered values are lagging. Lag makes the system slower.
A physical lowpass-filter could be used to reduce some of the noise going in to the Arduino, but other groups had tested that and reported it did not improve the result by much, if anything at all. The reason that a ordinary low pass filter does not work is because it draws current and disturbs the signal anyway. An active lowpass-filter however, could work because it does not draw any current away from the system.

In general there is a big disadvantage and reason to why not use digital filters instead of physical filters on analog signals. The speed of a digital filter is much slower than a psychical one and creates a latency to the process of the signal. Therefore a physical active lowpass filter might not just filter the signal better, but also increase the speed of the entire system. 

We wanted to achieve a system that acted as a "jack of all trades and master of none", but it was challenging to tune the PID controller. The system kept regulating even though the ball 
was at dead center. This is due to high noise in the sensor input. Even with the ball at setpoint, meaning the P-part is 0 and the I-part has fixed any steady-state, the D-part reacts small changes in the input which is actually just noise. The solution to the problem was to turn off the regulation when the ball had been inside the desired limits for one second.

There are many factors to take into consideration when choosing a sensor for the system. If we were to continue developing this system, a new sensor would be prioritized as well as implementing a physical active lowpass filter. 